\documentclass[hidelinks, 11pt, dvipsnames]{article}

\title{Gravitational N-body Problem}

\author{Kamalesh Paluru \\ Omer Sipra}

\usepackage[letterpaper,margin=0.7in]{geometry}

\geometry{
    top = 0.8in
}

\newcommand{\psection}[1]{{
    \begin{center}
        \noindent \rule{17cm}{0.4pt}
            \section*{\LARGE #1}
        \noindent \rule{17cm}{0.4pt}
    \end{center}
}}

\newcommand{\psubsection}[1]{
    \noindent 
    \section*{#1}
}

\begin{document}

\maketitle

\newpage

\psection{Proposal}

\psubsection{Abstract}
The $n$-body problem aims to solve for the motion of $n$ particles that influence each other gravitationally. The mutual influence(s) is determined by the gravitational law(s) that the system obeys. The 2-body problem has a complete analytic solution for classical systems, systems that abide by the rules of classical mechanics. However, things get more complicated for n $\ge$ 3, these only have analytic solutions for special cases and have to be approached numerically otherwise.

\psubsection{Strategy}
Since there is a range of laws and number of particles that we can pick from, we have candidates for tests and approximations. Assuming we have ``pre-requisite'' code to numerically solve a general $n$-body problem, we will approach the set of general $n$-body problems in the following order:
\begin{itemize}
    \item \textbf{Classical 2-body problem} \\ 
    \textsc{sanity check}: The easiest way to verify that our pre-requisite code works is to test it on well known 2-body systems. The Earth-Sun and Earth-Moon systems are good candidates since they have been extensively tested in the past (for launching satelites, etc). We can solve this problem \emph{analytically} and compare it to the results we get from our pre-requisite code (using integration methods). This mostly serves as a glorified sanity check.
    \item \textbf{Classical 3-body problem} \\ 
    \textsc{lagrange points}: As a continuation to the previous \textbf{test}, we will run our pre-requisite code for the analytically solveable restricted Earth-Sun-Satellite 3-body problem. We will also explore the Lagrange Points of this system and the behaviour of the satellite at and around these points. \\
    \textsc{scattering}: We will then look at a more general 3-body problem with a binary system and an incomming body.
    \item \textbf{Relativity and Quantum} \\
    Finally, we will increase the scope by considering other gravitational \emph{laws}. These are harder to solve for even two bodies so we will look at special cases of the 2-body problem and other cases that our pre-requisite code supports. \\
    \textsc{numerical relativity}: We will look at a Kepler problem (special 2-body problem with an infinitesimal mass, $m \ll M$) involving a black hole (or a similarly massive body, with mass $M$) and a photon (or a similarly infinitesimal mass, with some mass $m$). \\
    \textsc{few-body}: We will briefly explore how a two-body system behaves in the quantum realm. 
\end{itemize}

\end{document}
