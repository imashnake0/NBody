\documentclass[hidelinks, 11pt, dvipsnames]{article}

\title{Gravitational N-body Problem}

\author{Kamalesh Paluru \\ Omer Sipra}

\usepackage[letterpaper,margin=1in]{geometry}

\newcommand{\psection}[1]{{
    \begin{center}
        \noindent \rule{17cm}{0.4pt}
            \section*{\LARGE #1}
        \noindent \rule{17cm}{0.4pt}
    \end{center}
}}

\newcommand{\psubsection}[1]{
    \noindent 
    \section*{#1}
}

\begin{document}

\maketitle

\newpage

\psection{Proposal}

\psubsection{Abstract}
The $n$-body problem aims to solve for the motion of $n$ particles that influence each other gravitationally. The mutual influence(s) is determined by the gravitational law(s) that the system obeys. The 2-body problem has a complete analytic solution for classical systems, systems that abide by the rules of classical mechanics. However, things get more complicated for n $\ge$ 3, these only have analytic solutions for special cases and have to be approached numerically otherwise. \\

\psubsection{Strategy}

\end{document}
