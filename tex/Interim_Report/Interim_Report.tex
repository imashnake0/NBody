\documentclass[hidelinks, 11pt]{article}

\title{Gravitational N-body Problem}

\author{Kamalesh Paluru \\ Omer Sipra}

\usepackage[letterpaper,margin=0.7in]{geometry}
\usepackage{hyperref}

\hypersetup{
    colorlinks=true,
    linkcolor=blue,
    filecolor=magenta,      
    urlcolor=blue,
    pdftitle={Interim Report}
}


\geometry{
    top = 0.5in,
    right = 0.9in,
    left = 0.9in,
}

\newcommand{\psection}[1]{{
    \begin{center}
        \noindent \rule{17cm}{0.4pt}
            \section*{\LARGE #1}
        \noindent \rule{17cm}{0.4pt}
    \end{center}
}}

\newcommand{\psubsection}[1]{
    \noindent 
    \section*{#1}
}

\begin{document}

\maketitle

\newpage

\psection{Interim Report}

\psubsection{Classical 2-body Problem}

We decided to start with the most basic case: The classical 2-body problem. Once we have the necessary classes, we can make them more general to support more complicated systems. The two base classes are \texttt{Body} and \texttt{System}:

    \begin{itemize}
        \item \texttt{Body}: This class represents a body in our N-\texttt{body} system.
        \item \texttt{System}: This class takes a list of \texttt{Body} objects and contains methods to modify an instance of the system.
    \end{itemize}

\end{document}
