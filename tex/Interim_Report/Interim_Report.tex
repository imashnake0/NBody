\documentclass[hidelinks, 11pt]{article}

\title{Gravitational N-body Problem}

\author{Kamalesh Paluru \\ Omer Sipra}

\usepackage[letterpaper,margin=0.7in]{geometry}
\usepackage{hyperref}
\usepackage{amsmath}

\hypersetup{
    colorlinks=true,
    linkcolor=blue,
    filecolor=magenta,      
    urlcolor=blue,
    pdftitle={Interim Report}
}


\geometry{
    top = 0.5in,
    right = 0.9in,
    left = 0.9in,
}

\newcommand{\psection}[1]{{
    \begin{center}
        \noindent \rule{17cm}{0.4pt}
            \section*{\LARGE #1}
        \noindent \rule{17cm}{0.4pt}
    \end{center}
}}

\newcommand{\psubsection}[1]{
    \noindent 
    \section*{#1}
}


\newcommand{\psubsubsection}[1]{
    \noindent
    \subsection*{#1}
}

\begin{document}

\maketitle

\psubsection{Summary of Contributions}
\psubsubsection{Omer Sipra}
\begin{itemize}
    \item Worked primarily on documentation and theory for the N-body problem.
\end{itemize}

\psubsubsection{Kamalesh Paluru}
\begin{itemize}
    \item Implemented the theory and created a program to simulate an N-body system.
\end{itemize}
\newpage

\psection{Interim Report}

\psubsection{Classical 2-body Problem}

We decided to start with the most basic case: The classical 2-body problem. Once we have the necessary classes, we can make them more general to support more complicated systems. The two base classes are \texttt{Body} and \texttt{System}:

    \begin{itemize}
        \item \texttt{Body}: This class represents a body in our N-\texttt{body} system.
        \item \texttt{System}: This class takes a list of \texttt{Body} objects and contains methods to modify an instance of the system (and the bodies in it).
    \end{itemize}

\noindent
Using OOP made it much easier to create a strong foundation and will help make it easier to generify the program.

    \psubsubsection{\texttt{System.simulate(until=time)}}

    \begin{itemize}
        \item This method mutates the body object in \texttt{System}, it hence returns the updated positions and velocities of the bodies.
        \item The equations used for the classical 2-body system are relatively simple, \texttt{simulate} algorithmically updates the positions and velocities of each body:
        \begin{itemize}
            \item Repeats the following steps for every time differential in a given time interval:
            \item Update the velocity of every body using:
            \[ \vec{v}_f = \vec{v}_i + d\vec{v} \]
            \[ \vec{v}_f = \vec{v}_i + \vec{a} \cdot dt \]
            \[ \vec{v}_f = \vec{v}_i + \frac{\vec{F}_\text{net}}{m} \cdot dt \]
            Here; $m$ is the mass of the body, $\vec{v}_i$ is the initial velocity, $\vec{F}_\text{net}$ is initially calculated by the constructor (and later updated) when \texttt{System} is initialized, and $dt$ is the given time differential.
            \item Update the position of every body using:
            \[ \vec{x}_f = \vec{x}_i + d\vec{x} \]
            \[ \vec{x}_f = \vec{x}_i + \vec{v}_i \cdot dt \]
            \item Tick the clock:
            \[ t_f = t_i + dt \]
            \item Finally, we update the net force on each body using the updated positions and velocities.
        \end{itemize}
    \end{itemize}

    \psubsection{Validation}

    \begin{itemize}
        \item Unfortunately, the position of \texttt{earth} doesn't have the expected sequence of values. We highly suspect there is something wrong with the way the position/velocities are updated. \texttt{sun} however does seem to stay close to the origin as expected.
        \item Although we can't produce plots just yet (for the reason above), we have a strong foundation to build upon with these base classes.
    \end{itemize}
\end{document}
